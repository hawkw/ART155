\documentclass[a4paper,12pt]{article}
\usepackage{url}
\usepackage{setspace}
\title{\vspace{-4cm}ART155: Response to \textit{Disrupting~Assumptions}}
\author{Eliza Weisman}
\begin{document}
\maketitle
\doublespacing

The most salient point put forwards in \textit{Disrupting Assumptions} is that much of Charles Ray's work challenghes the audience's perception of reality. In a manner similar to optical illusions such as the `devil's fork` drawing, or the photograph of the dress which recently took the internet by storm by appearing to be different colors to different viewers, pieces like \textit{32~x~33~x~35~=~34~x~33~x~35}, \textit{Yes}, and \textit{Fall '91} seem calculated to cause museum-goers audience to question their senses.

By troubling the veracity of sense data, as Charles Ray does, the artist erodes the philosophical underpinnings of empiricism, drawing the audience, perhaps unwillingly, into the very same debate presented by Socrates in the prior reading assignment. The suggestion that the audience cannot trust what they see or touch makes the implicit statement that there is no such thing as objective reality.

An attack on the concept of the `real' is only the first of many possible meanings that we can draw out of Ray's twisting of his audience's perceptions, however. As the author of \textit{Disrupting Assumptions} observes in their discussion of Ray's \textit{Yes}, works of art which challenge their audience's experience of reality can often place viewers in altered state not unlike those achieved through the use of drugs. It's interesting to note that the use of altered states, whether achieved through drugs, meditation, or other psychonautic techniques, is quite common in the religious or magical practices of almost every culture. In fact, many halluciogenic drugs are referred to by anthropologists as `entheogens' --- in the Latin, ``that which creates [the experience of] the divine''. In religions such as Buddhism, for example, the alterned state brought on by meditation is viewed as a heightened form of conciousness, while in some modern occult traditions, it is supposed that the mind in an altered state has the power to impose its will upon reality.

Ray himself has stated that many of his works, including \textit{Yes}, are informed directly by his experiences of ceremonial chemicals such as LSD, so it isn't terribly difficult for us to draw this comparison. However, I feel as though it's quite easy to dismiss works of this nature as being `about drugs' without considering the importance of altered states in religion and mysticism. From this perspective, artwork which confuses the senses becomes less like an attack on the viewer, and more like a revealing, an apocalypse in the original sense of the word. The artist is no longer shocking the audience or disrupting their experience of reality, but acting as a guide or spiritual teacher; bringing the viewer to a higher state of conciousness.

For all the pretension of the previous few paragraphs, however, it's important to also note that it's quite easy for this sort of artwork to fall flat or appear trite. If the entire nature of a piece is that it surprises the audience by challenging their perceptions, all it takes is an off-hand comment --- ``Charles Ray's \textit{Fall
'91} is bigger than it appears to be!'' --- or remark in gallery material to spoil the realisation completely. If the piece has no additional meaning in its semiotic payload, then the entire work is ruined completely for an individual viewer, conveying nothing. It's important, I think, that this sort of work use the altered states that it might evoke to explore some concept or symbol, rather than simply to trick the viewer's senses just \textit{because}.
\end{document}
