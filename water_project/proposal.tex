\documentclass[a4paper,nobib]{tufte-handout}
\usepackage{url}
\usepackage{setspace}

\usepackage[english]{babel}
\usepackage{csquotes}

\usepackage{hyperref}
\usepackage{cleveref}

\usepackage{graphicx} % allow embedded images
  \setkeys{Gin}{width=\linewidth,totalheight=\textheight,keepaspectratio}
  \graphicspath{{../imgs/}} % set of paths to search for images
\usepackage{amsmath}  % extended mathematics
\usepackage{booktabs} % book-quality tables
\usepackage{units}    % non-stacked fractions and better unit spacing
\usepackage{multicol} % multiple column layout facilities
% \usepackage{biblatex}
% \usepackage[ backend=bibtex
%            , sortcites=true
%            , sorting=nty
%            , backref
%            , natbib
%            , hyperref
%            ]{biblatex-chicago}
\usepackage[ autocite  = footnote
           , backend   = bibtex
           , sortcites = true
           , sorting   = nty
           , backref
           , natbib
           , hyperref
           ]{biblatex-chicago}
%
% \usepackage[ backend=bibtex
%            , sortcites=true
%            , sorting=nty
%            , backref
%            , natbib
%            , hyperref
%            ]{biblatex-chicago}
\bibliography{proposal.bib}
\title{\textbf{ART155 Project Proposal:} Water of Life}
\author{Eliza Weisman}
\begin{document}
\maketitle
\doublespacing

Water is life. It is necessary for all living things, whether plants or animals. But what many take for granted is that water, the substance itself, more often than not literally \textit{is} life: even the clearest water often teems with untold millions of micro{\"o}rganisms, invisible to the naked eye. These microbiota can sometimes be toxic, but they are just as likely to be invaluable to human life, and frequently have little to no impact on us whatsoever. As humans, we interact with these forms of life on a daily basis, although generally without noticing; the duality of this interaction is striking, as bacteria are both a leading cause of death and necessary for our lives.

While the size of these life-forms makes them imperceptable to humans, we sometimes can witness the effects of their actions on the substrate that they inhabit. The toxic algal blooms of red tide, when tiny, poisonous plants cluster so tightly that the water appears blood-red, is perhaps the most famous. Rightly so, for its effects can be quite deadly. Much more bizarre and fascinating, however, are the effects of bioluminescent algae and bacteria, which are endowed with the unique capability of emitting light. The ghostly blue glow produced by these organisms' metabolic processes is a breathtaking sight that provides a powerful illustration of the invisible world surrounding us.

Artwork that deals with environmental issues usually makes an attempt to draw attention to a specific issue or problem. While this sort of work is frequently effective in influencing or inciting public discourse surrounding that topic, I personally feel that it tends to fall flat in its attept to create an aesthetic experience. This sort of work, which blurs the boundary between art and activism, more often than not feels like it is only activism; effective methods of public communication which appear trite when viewed from an artistic standpoint. Activism need not be subtle, but good works of art ought not spell out everything for their audiences.

Considering both this observation and the tremendous importance of environmental health issues, I aim personally to create works which impress upon their audience the grandeur, beauty, and importance of nature, and the complex role humans inhabit within it. Rather than call attention to a particular environmental crisis, my work would ideally induce the audience to seek out that information from other sources by engaging with the concept of human interactions with nature in a more general, symbolic manner.

Mystical or occult imagery has become a recurring theme in my artistic practice, as well. While I am largely non-religions, I find that these symbols hold a great deal of semiotic power, and are deeply meaningful both culturally and psychologically. To some extent, it is possible to hold a rational or scientific worldview and still engage in mystical practices. The ability of new-age or occult practices to shape and improve the individual's life is often dismissed as the placebo effect, but it is worth acknowledging that the placebo effect and belief in general are actually quite powerful. If the placebo effect, and belief in general, were not capable of significantly impacting our lives, it would not be necessary to control for it in drug trials; that is to say, the placebo effect \textit{matters}.

Some modern forms of occult practice, such as Chaos magick~\autocite{carroll1987liber,carroll1992liber,sherwin1992book}, acknowledge the power of belief, coming close to the acknowledgement that magic is essentially the application of the placebo effect. Chaos magick claims that the human will, when empowered by belief, is capable of shaping reality; from a postmodernist perspective, whether this is because the practitioner is shaping the world or their own subconcious doesn't particularly matter. The salient question, then, is not whether or not magic is real, but whether its practice is \textit{useful} to the individual; a perspective that seems rather similar to the way scientific theories are generally assessed in the post-Karl Popper~\autocite{popper1999all} world.

Bioluminescent bacteria such as \textit{Aliivibrio fischeri}~\autocite{brock,dictPhotobacterium} and \textit{Photobacterium phosphoreum}~\autocite{dictPhotobacterium} have the potential to be powerful symbols. The light emitted by these bacteria has a particular magic or sense of wonder posessed by few other natural phenomena. The vessels in which these organisms could be cultured, such as agar plates and laboratory glassware, evoke human methods of interacting with other parts of nature. Beakers, flasks, and test tubes also seem to occupy a curious semiotic space between science and magic --- they evoke both the scientist's laboratory and the witch's workshop, causing the viewer to wonder whether they are observing a chemical experiment or a magical potion. As scientific apparatus, these objects are designed with only practical considerations, they are not conventionally beautiful. Their use in an artistic work engages with the boundary between functional objects and those that serve an aesthetic purpose. Furthermore, it is interesting to note that these luminescent organisms are biologically quite similar to the bacteria responsible for cholera and other forms of waterbourne illnesses~\autocite{dictVibrio}. As symbols, these bacteria and their methods of cultivation draw out a number of fascinating dualities: between science and magic, life and death, form and function.

Fortunately for the artist who wishes to harness the luminous display of these single-celled performers, organisms such as \textit{A. fischeri}, \textit{P. phosphoreum}, and the bioluminescent algae of the phylum \textit{Dinoflagellata} are relatively easy to culture, requiring only a small amount of laboratory equipment. The number of informal tutorials available on the internet~\autocite{instructablesLightbulb,instructablesAlgae,instructablesFountain} provides a strong testimony demonstrating the ease of working with such organisms.

The use of micro{\"o}rganisms and other living things in art-making is not at all unheard-of; in fact, it is a major component of an emerging artistic movement referred to as ``BioArt'', in which living organisms and biological processes are used as artistic media~\autocite{anker2004molecular,pentecost2008outfitting}. A number of artists working in this movement have worked with bioluminescence, such as Eduardo Kac's ``GFP Bunny''~(2003), in which a rabbit was genetically engineered to produce a flourescent protein~\autocite{kac2003gfp}; ``Bioglyphs''~(2003), a collaborative project created by artists and scientists at Montana State University and the Center for Biofilm Engineering~\autocite{petkewich2003scientist,bioglyphs}; and Byron Rich's ``Benign Nor Hostile, Merely Indifferent'' (2013)~\autocite{benignNorHostile}. In the lexicon of popular culture, bioluminescence speaks strongly of the of the powerful and unnatural; in films, comic books, and video games, monsters and superhuman characters created through genetic tampering or mutation are frequently marked by an actinic glow. Many of these artists may be playing with the connotations of bioluminescence created by the popular media. For example, the bright green flourescence of Alba, Eduardo Kac's luminescent rabbit, is remarkably similar to \textit{Superman}'s kryptonite or the complexion of Marvel Comics' \textit{Incredible Hulk}.

Other artists have worked with occult themes, as well. One of the most interesting is Joshua Madara, a digital and electronic artist. Madara considers himself an occultist or thaumaturge rather than an artist, stating that his works are created for magical rather than aesthetic purposes, but concludes that others may label him as an artist~\autocite{madara}. Madara's works involve both technology and the occult, and trouble the strict delineation between technology, art, and mystical practice, a subject which he has written about at length~\autocite{madarabeing,madaramagic}. While I approach my work as an artist and not as an occult practitioner, as Madara does, I intend for my work to create a spiritual or mystical experience for aesthetic purposes, borrowing the psychological as well as symbolic language of the occult.

My planned piece for the Water Exhibition will consist of a number of vessels of varying forms (such as plastic water bottles, laboratory glassware, etc), filled with water containing colonies of luminescent micro{\"o}rganisms and possibly other fluids. These vessels will be arranged into a kinetic sculpture in which fluids are pumped between containers, stirred, or shaken, and components of the assemblage rotated or otherwise moved, in time with natural cycles calculated by a microcontroller such as a Raspberry Pi. This motion may be synchronized with the human heartbeat, the tides, the motion of the sun, moon, and other astronomical objects, and cyclical processes within the human body. The piece may also incorporate the use of symbols of mystical importance in various traditions or belief systems. In addition, I intend to document the stages of the work's creation involving the culturing and manipulation of micro{\"o}rganisms; this activity could include performative elements that may potentially be worked into the finished piece.

\pagebreak
\begingroup
\singlespacing
\setlength{\emergencystretch}{1.5em} % this fixes overfull hboxes in citations
                                   % (according to the interwebs)
\printbibliography
\endgroup

\end{document}
