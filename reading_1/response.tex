\documentclass[a4paper,12pt]{article}
\usepackage{url}
\usepackage{setspace}
\title{ART155: Response to Plato's Allegory~of~the~Cave}
\author{Eliza Weisman}
\begin{document}
\maketitle
\doublespacing

In the Allegory of the Cave, Plato illustrates the concepts of his theory of Forms; an epistemological worldview wherein the essential qualties of objects exist independantly from them. These ``forms'', such as the form of redness, or beauty, or cheese, exist independantly from the physical things that embody them, such as a red tomato, a beautiful sunset, or a wheel of gouda. While any individual piece of cheese may be distinct other cheeses in any number of ways, it nonetheless participates in the essential essence of cheesiness or embodies the cheese-nature.

% To me, the obvious analogue to Plato's theory of forms in modern thought is the concept of object-oriented programming in computer science. I have always held to the conceit that --- to resort to an SAT-style analogy --- epistemology is to computer science much as physics is to engineering, so perhaps it is a legitimate comparison.
% %
% The object-oriented programming paradigm models systems through the interaction of \textit{objects}, which bundle together information or data with the behaviors or operations that can be performed upon it. Each object is a member of one or more \textit{classes}, which describe the data stored in that type of object and the behaviors that it can carry out. In the object-oriented model, an object is said to \textit{instantiate} the class that it is a member of. These classes can be arranged into a hierarchy wherein some classes are subclasses of others; this is called inheritance. For example, in a hypothetical course-registration system, the object representing Byron might be an instance of \texttt{Professor}, while the object representing Eliza might be an instance of \texttt{Student}. The \texttt{Student} and \texttt{Professor} classes could both be subclasses of \texttt{Person}, which might define information such as a first name, last name, or telephone number. Since it doesn't make sense to track a \texttt{Professor}'s major, minor, or GPA, that information might all be defined in \texttt{Student}, while \texttt{Professor} might include data such as an office number or a list of classes taught. The parallels between this way of viewing the world and Plato's forms seems readily app arent.
%
% But why draw this comparison? Why does it matter that the ideas of Plato would be reinvented by modern computer programmers? I think that this observation is important because it speaks to both the strengths and to the weaknesses of the Platonic worldview. Scientists once believed that their theories could make true claims about reality, but the work of philosophers like Karl Popper and David Hume ruined that dream by demonstrating that an inductive theory can be proven wrong but can never be proven right. Now, scientific theories are assessed in part by how \textit{useful} they are --- we cannot ever really prove that the theory of electromagnetism, for instance, actually describes the way the world works, but it certainly is useful in that its predictions do permit us to make things like toasters and lightbulbs. It seems to me that we ought to be able to play a similar game with the theories of philosophy, but so rarely do we see them have direct applications.

% How does the mind find meaning in an object-oriented world? In a computer program, knowledge is declarative –-- the programmer explicitly states the types of things to the machine --- but in the real world, we have no such luxury.
How does the mind find meaning in such a world? We must guess at the inner essences of things from observations of their outward characteristics, just like the prisoners in Socrates' cave watching shadows on the wall. Fortunately, to assist us in this difficult task, we are equipped with the most powerful pattern-recognition engine in the known universe: the human mind. We are quite good, I've found, at this sort of learning, watching for outward evidence of the internal states or natures of things. Perhaps too good, even, for we often see patterns and connections where there are none, the force that powers conspiracy theories and the finding of faces in things like tortillas and the Moon.

This Platonic picture of learning seems like a reasonable theory of knowledge, but the work of later philosophers like David Hume and Karl Popper trouble it significantly. Scientists once believed that their theories could make true claims about reality, but Popper ruined that beautiful dream with his idea that an inductive theory can be proven wrong but can never be proven right. It seems to me that while learning does function by observing outward appearances, the brain uses these signifiers to construct an \textit{internal} mental model of, say, redness, rather than deriving the true form of redness that's hanging out in the ether with the perfect circle your math teacher couldn't draw and the line that actually goes off infinitely instead of terminating sadly with a messy little arrow. The forms, I suppose, do exist, but from my perspective, it seems much more likely that they only exist when reified in the mind of each individual observer.

If we continue along Plato's line of thinking, then it seems apparent that art-making is perhaps a hijacking of this cognitive process; or, perhaps, a natural consequence of it. The artist marinates themself in signifiers and then uses them to produce thoughts or emotional states in the audience, without the stimulus that actually `back' those thoughts; like smoke without fire.re

This conception of art-making seems somewhat lacking, however. If we suppose that their are no true forms, like Plato seems to believe, but only human-constructed prediction models, then really, what the artist is doing is sharing bits of their mental model of the world and hoping others react to those symbols in analoguous ways.

Nonetheless, even this version of the act of making art is still incomplete; it fails to consider that meaning can often be --- and usually is --- assigned to things by culture, rather than through learning from the environment. This is rather lucky for us as humans, since it would take us some time to learn empiricially all of the knowledge that is stored in culture! Culture, though, also contains a number of symbol $\rightarrow$ object mappings that are not, strictly speaking, knowledge, and it is often these associations that are the most powerful for us. The modern conceptual artist tends to work more with abstractions plucked from the common subconcious of culture than with their own mental map of the physical world.

\end{document}
